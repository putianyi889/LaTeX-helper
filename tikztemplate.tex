\usepackage{tikz}
\usetikzlibrary{calc}

\newcommand{\drawquasimatrix}[2]{
    \draw (0,0) rectangle (#2,#1);
    \foreach \x in {0.5,...,#2} {
        \draw (\x,0.5) -- (\x,#1-0.5);
    }
}

\newcommand{\drawinfquasimatrix}[2]{
    \draw (0,0) rectangle (#2,#1);
    \foreach \x in {1.5,...,#2} {
        \draw (\x-1,0.5) -- (\x-1,#1-0.5);
    }
    \node at (#2-0.5,#1/2) {$\cdots$};
}

\newcommand{\drawmatrix}[2]{
    \draw (0,0) rectangle (#2,#1);
    \foreach \x in {0.5,...,#2} {
        \foreach \y in {0.5,...,#1} {
            \fill (\x,\y) circle (2pt);
        }
    }
}

\newcommand{\drawinfvector}[1]{
    \draw (0,0) rectangle (1,#1);
    \foreach \y in {1.5,...,#1} {
        \fill (0.5,\y) circle (2pt);
    }
    \node[anchor=mid] (A) at (0.5,0.5) {$\vdots$};
}

\newcommand{\drawdensematrix}[2]{
    \foreach \m in {1,...,#1} {
        \foreach \n in {1,...,#2} {
            \draw (\n,-\m) circle (0.1);
        }
    }
}

% height, width, lower band, upper band
\newcommand{\drawbandedmatrix}[4]{%
    \foreach \m in {1,...,#1} {%
        \foreach \n [evaluate={\n+\m} as \neval] in {-#3,...,#4} {%
            \pgfmathparse{(\neval < 1 ? 0 :
                      (\neval > #2 ? 0 : 1))}
            \ifnum\pgfmathresult=1
                \draw (\neval,-\m) circle (0.1);
            \fi
        }
    }
}

\newcommand{\drawmatrixbox}[2]{
    \draw (0.5,-0.5) rectangle (#2+0.5, -#1-0.5);
}

% credit: https://tex.stackexchange.com/a/354243/199226
% \drawloop[options]{node}{start angle}{end angle} ... ;
\usetikzlibrary{calc}
\tikzset{stretch/.initial=1}
\newcommand\drawloop[4][]%
   {\draw[shorten <=0pt, shorten >=0pt,#1]
      ($(#2)!\pgfkeysvalueof{/tikz/stretch}!(#2.#3)$)
      let \p1=($(#2.center)!\pgfkeysvalueof{/tikz/stretch}!(#2.north)-(#2)$),
          \n1= {veclen(\x1,\y1)*sin(0.5*(#4-#3))/sin(0.5*(180-#4+#3))}
      in arc [start angle={#3-90}, end angle={#4+90}, radius=\n1]%
   }