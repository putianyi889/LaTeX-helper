\usepackage{mathtools}
\usepackage{amsmath}
\usepackage{amsfonts}

\newcommand{\Beta}{\mathrm{B}} % Upper beta
\DeclareMathOperator{\sign}{sgn} % sign function
\DeclareMathOperator{\diag}{diag} % diagonal matrix
\DeclareMathOperator{\spanspace}{span}
\DeclareMathOperator{\supp}{supp} % function support
\DeclareMathOperator{\identity}{id} % identity operator
\DeclareMathOperator{\Li}{Li} % polylogarithm function
\DeclareMathOperator{\erfc}{erfc} % complementary error function
\DeclareMathOperator{\sech}{sech} % hyperbolic secant
\DeclareMathOperator{\csch}{csch} % hyperbolic cosecant
\DeclareMathOperator{\arsinh}{arsinh} % inverse hyperbolic sine
\DeclareMathOperator{\arcosh}{arcosh} % inverse hyperbolic cosine
\DeclareMathOperator{\artanh}{artanh} % inverse hyperbolic tangent
\DeclareMathOperator{\arcoth}{arcoth} % inverse hyperbolic cotangent
\DeclareMathOperator{\arsech}{arsech} % inverse hyperbolic secant
\DeclareMathOperator{\arcsch}{arcsch} % inverse hyperbolic cosecant
\AtBeginDocument{\renewcommand{\d}{\:\mathrm{d}}} % differential symbol
\newcommand{\pFq}[2]{{}_{#1}F_{#2}} % hypergeometric function
\newcommand{\half}[1][1]{\frac{#1}{2}} % lazy wrapping
\newcommand{\f}[1]{\!\left({#1}\right)\!} % lazy wrapping for ()
\newcommand{\abs}[1]{\left\lvert{#1}\right\rvert} % absolute value
\newcommand{\floor}[1]{\left\lfloor{#1}\right\rfloor} % rounding down
\newcommand{\ceil}[1]{\left\lceil{#1}\right\rceil} % rounding up
\newcommand{\tfloor}[1]{\lfloor{#1}\rfloor} % rounding down with tiny delimiter
\newcommand{\tceil}[1]{\lceil{#1}\rceil} % rounding up with tiny delimiter
\newcommand{\Atop}[2]{\genfrac{}{}{0pt}{}{#1}{#2}} % safe wrapping of \atop
\newcommand{\hypergeometric}[5]{\pFq{#1}{#2}\left(\Atop{#3}{#4};{#5}\right)} % hypergeometric function
\newcommand{\lazysplit}[1]{\[\begin{split}#1\end{split}\]}
\newcommand{\innerproduct}[1]{\left\langle{#1}\right\rangle} % inner product

% principal value integral
% credit: https://tex.stackexchange.com/a/760/199226
\def\Xint#1{\mathchoice
{\XXint\displaystyle\textstyle{#1}}%
{\XXint\textstyle\scriptstyle{#1}}%
{\XXint\scriptstyle\scriptscriptstyle{#1}}%
{\XXint\scriptscriptstyle\scriptscriptstyle{#1}}%
\!\int}
\def\XXint#1#2#3{{\setbox0=\hbox{$#1{#2#3}{\int}$ }
\vcenter{\hbox{$#2#3$ }}\kern-.6\wd0}}
\def\ddashint{\Xint=}
\def\dashint{\Xint-}

% combine multicols and subequations environments
% credit: https://tex.stackexchange.com/a/513552/199226
\newenvironment{multicolsubequations}[1]{
    \begin{subequations}
        \setlength{\abovedisplayskip}{0pt}
        \setlength{\belowdisplayskip}{0pt}
        \begin{multicols}{#1}
            \vspace*{-2\baselineskip}
}{
        \end{multicols}
    \end{subequations}
}

% add an extra tag to the left of an equation
% credit: https://tex.stackexchange.com/a/664316/199226
\makeatletter
\def\ltag#1{%
\stepcounter{equation}%
\tag*{}%
\def\df@tag{(\theequation)\llap{\rlap{#1}\hspace{\columnwidth}}}%
}
\makeatother