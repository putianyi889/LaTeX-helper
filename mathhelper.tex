\usepackage{mathtools}
\usepackage{amsmath}
\usepackage{tikzpagenodes}

\newcommand{\Beta}{\mathrm{B}} % Upper beta
\DeclareMathOperator{\sign}{sgn} % sign function
\DeclareMathOperator{\diag}{diag} % diagonal matrix
\DeclareMathOperator{\Li}{Li} % polylogarithm function
\AtBeginDocument{\renewcommand{\d}{\:\mathrm{d}}} % differential symbol
\newcommand{\pFq}[2]{{}_{#1}F_{#2}} % hypergeometric function
\newcommand{\half}[1][1]{\frac{#1}{2}} % lazy wrapping
\newcommand{\f}[1]{\!\left({#1}\right)\!} % lazy wrapping for ()
\newcommand{\abs}[1]{\left\lvert{#1}\right\rvert} % absolute value
\newcommand{\Atop}[2]{\genfrac{}{}{0pt}{}{#1}{#2}} % safe wrapping of \atop
\newcommand{\hypergeometric}[5]{\pFq{#1}{#2}\left(\Atop{#3}{#4};{#5}\right)} % hypergeometric function
\newcommand{\lazysplit}[1]{\[\begin{split}#1\end{split}\]}
\newcommand{\innerproduct}[1]{\left\langle{#1}\right\rangle} % inner product

% principal value integral
% credit: https://tex.stackexchange.com/a/760/199226
\def\Xint#1{\mathchoice
{\XXint\displaystyle\textstyle{#1}}%
{\XXint\textstyle\scriptstyle{#1}}%
{\XXint\scriptstyle\scriptscriptstyle{#1}}%
{\XXint\scriptscriptstyle\scriptscriptstyle{#1}}%
\!\int}
\def\XXint#1#2#3{{\setbox0=\hbox{$#1{#2#3}{\int}$ }
\vcenter{\hbox{$#2#3$ }}\kern-.6\wd0}}
\def\ddashint{\Xint=}
\def\dashint{\Xint-}

% combine multicols and subequations environments
% credit: https://tex.stackexchange.com/a/513552/199226
\newenvironment{multicolsubequations}[1]{
    \begin{subequations}
        \setlength{\abovedisplayskip}{0pt}
        \setlength{\belowdisplayskip}{0pt}
        \begin{multicols}{#1}
            \vspace*{-2\baselineskip}
}{
        \end{multicols}
    \end{subequations}
}

% add an extra tag to the line
% credit: https://tex.stackexchange.com/a/574795/199226
\newcommand{\rtag}[1]{
    \begin{tikzpicture}[baseline=(tmp.base),remember picture]
        \node[inner sep=0pt](tmp){\vphantom{1}};
        \begin{scope}[overlay]
            \path (current page text area.east|-tmp.base)
                node[anchor=base east,inner sep=0pt,outer sep=0pt]{#1};
        \end{scope}
    \end{tikzpicture}
}
\newcommand{\ltag}[1]{
    \begin{tikzpicture}[baseline=(tmp.base),remember picture]
        \node[inner sep=0pt](tmp){\vphantom{1}};
        \begin{scope}[overlay]
            \path (current page text area.west|-tmp.base)
                node[anchor=base west,inner sep=0pt,outer sep=0pt]{#1};
        \end{scope}
    \end{tikzpicture}
}