\usepackage{mathtools}
\usepackage{amsmath}

\newcommand{\Beta}{\mathrm{B}} % Upper beta
\DeclareMathOperator{\sign}{sgn} % sign function
\DeclareMathOperator{\Li}{Li} % polylogarithm function
\AtBeginDocument{\renewcommand{\d}{\:\mathrm{d}}} % differential symbol
\newcommand{\pFq}[2]{{}_{#1}F_{#2}} % hypergeometric function
\newcommand{\half}[1][1]{\frac{#1}{2}} % lazy wrapping
\newcommand{\f}[1]{\!\left({#1}\right)\!} % lazy wrapping
\newcommand{\abs}[1]{\left\lvert{#1}\right\rvert} % absolute value
\newcommand{\Atop}[2]{\genfrac{}{}{0pt}{}{#1}{#2}} % safe wrapping of \atop
\newcommand{\hypergeometric}[5]{\pFq{#1}{#2}\left(\Atop{#3}{#4};{#5}\right)} % hypergeometric function

% principal value integral
% credit: https://tex.stackexchange.com/a/760/199226
\def\Xint#1{\mathchoice
{\XXint\displaystyle\textstyle{#1}}%
{\XXint\textstyle\scriptstyle{#1}}%
{\XXint\scriptstyle\scriptscriptstyle{#1}}%
{\XXint\scriptscriptstyle\scriptscriptstyle{#1}}%
\!\int}
\def\XXint#1#2#3{{\setbox0=\hbox{$#1{#2#3}{\int}$ }
\vcenter{\hbox{$#2#3$ }}\kern-.6\wd0}}
\def\ddashint{\Xint=}
\def\dashint{\Xint-}
