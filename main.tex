\documentclass{article}
\usepackage[utf8]{inputenc}
\usepackage{url}

\usepackage{mathtools}
\usepackage{amsmath}

\newcommand{\Beta}{\mathrm{B}} % Upper beta
\DeclareMathOperator{\sign}{sgn} % sign function
\DeclareMathOperator{\Li}{Li} % polylogarithm function
\AtBeginDocument{\renewcommand{\d}{\:\mathrm{d}}} % differential symbol
\newcommand{\pFq}[2]{{}_{#1}F_{#2}} % hypergeometric function
\newcommand{\half}[1][1]{\frac{#1}{2}} % lazy wrapping
\newcommand{\f}[1]{\!\left({#1}\right)\!} % lazy wrapping
\newcommand{\abs}[1]{\left\lvert{#1}\right\rvert} % absolute value
\newcommand{\Atop}[2]{\genfrac{}{}{0pt}{}{#1}{#2}} % safe wrapping of \atop
\newcommand{\hypergeometric}[5]{\pFq{#1}{#2}\left(\Atop{#3}{#4};{#5}\right)} % hypergeometric function

% principal value integral
% credit: https://tex.stackexchange.com/a/760/199226
\def\Xint#1{\mathchoice
{\XXint\displaystyle\textstyle{#1}}%
{\XXint\textstyle\scriptstyle{#1}}%
{\XXint\scriptstyle\scriptscriptstyle{#1}}%
{\XXint\scriptscriptstyle\scriptscriptstyle{#1}}%
\!\int}
\def\XXint#1#2#3{{\setbox0=\hbox{$#1{#2#3}{\int}$ }
\vcenter{\hbox{$#2#3$ }}\kern-.6\wd0}}
\def\ddashint{\Xint=}
\def\dashint{\Xint-}

\usepackage{amsthm}
\newtheorem{theorem}{Theorem}
\newtheorem{corollary}{Corollary}
\newtheorem{proposition}{Proposition}
\newtheorem{lemma}{Lemma}
\newtheorem{conjecture}{Conjecture}

\theoremstyle{remark}
\newtheorem{remark}{Remark}

\theoremstyle{definition}
\newtheorem{definition}{Definition}
\newtheorem{example}{Example}
\newtheorem{calculation}{Calculation}
\input{unicode}
\usepackage{xcolor}
\newcommand{\nightmode}{
    \pagecolor{black}
    \color{gray}
    \usepackage[font={color=gray}]{caption}
}

\usepackage{datetime2}
\usepackage{xstring}
% if the time is not between #1:00 and #2:00 UTC then use nightmode
\newcommand{\smartnightmode}[2]{
    \StrBefore{\DTMcurrenttime}{:}[\hour]
    \ifnum\hour<#2
        \ifnum\hour<#1
            \nightmode
        \fi
    \else
        \nightmode
    \fi
}
\usetikzlibrary{calc}

\newcommand{\drawquasimatrix}[2]{
    \draw (0,0) rectangle (#2,#1);
    \foreach \x in {0.5,...,#2} {
        \draw (\x,0.5) -- (\x,#1-0.5);
    }
}

\newcommand{\drawinfquasimatrix}[2]{
    \draw (0,0) rectangle (#2,#1);
    \foreach \x in {1.5,...,#2} {
        \draw (\x-1,0.5) -- (\x-1,#1-0.5);
    }
    \node at (#2-0.5,#1/2) {$\cdots$};
}

\newcommand{\drawmatrix}[2]{
    \draw (0,0) rectangle (#2,#1);
    \foreach \x in {0.5,...,#2} {
        \foreach \y in {0.5,...,#1} {
            \fill (\x,\y) circle (2pt);
        }
    }
}

\newcommand{\drawinfvector}[1]{
    \draw (0,0) rectangle (1,#1);
    \foreach \y in {1.5,...,#1} {
        \fill (0.5,\y) circle (2pt);
    }
    \node[anchor=mid] (A) at (0.5,0.5) {$\vdots$};
}

\title{Preambles}
\author{Tianyi Pu}
\date{\today}

\begin{document}

\maketitle

\section{Math Commands}
\begin{itemize}
    \item[\texttt{\textbackslash abs[1]}] $\abs{\#1}$, the absolute value.
    \item[\texttt{\textbackslash arcosh}] $\arcosh$, the inverse hyperbolic cosine.
    \item[\texttt{\textbackslash arcoth}] $\arcoth$, the inverse hyperbolic cotangent.
    \item[\texttt{\textbackslash arcsch}] $\arcsch$, the inverse hyperbolic cosecant.
    \item[\texttt{\textbackslash arsech}] $\arsech$, the inverse hyperbolic secant.
    \item[\texttt{\textbackslash arsinh}] $\arsinh$, the inverse hyperbolic sine.
    \item[\texttt{\textbackslash artanh}] $\artanh$, the inverse hyperbolic tangent.
    \item[\texttt{\textbackslash Atop[2]}] $\Atop{\#1}{\#2}$, a safe wrapping of \texttt{\textbackslash atop}, the line-less fraction.
    \item[\texttt{\textbackslash Beta}] $\Beta$, the big $\beta$.
    \item[\texttt{\textbackslash ceil[1]}] $\ceil{\#1}$, rounding up.
    \item[\texttt{\textbackslash csch}] $\csch$, the hyperbolic cosecant.
    \item[\texttt{\textbackslash d}] $\d$, the differential operator.
    \item[\texttt{\textbackslash diag}] $\diag$, a diagonal matrix.
    \item[\texttt{\textbackslash erfc}] $\erfc$, the complementary error function.
    \item[\texttt{\textbackslash f[1]}] $\f{\#1}$, a lazy wrapping of \texttt{\textbackslash left(\#1\textbackslash right)}.
    \item[\texttt{\textbackslash floor[1]}] $\floor{\#1}$, rounding down.
    \item[\texttt{\textbackslash half[1][1]}] $\half[\#1]$, a lazy wrapping of halves.
    \item[\texttt{\textbackslash hypergeometric[5]}] $\hypergeometric{\#1}{\#2}{\#3}{\#4}{\#5}$, the hypergeometric function with arguments.
    \item[\texttt{\textbackslash hypergeometricOlver[5]}] $\hypergeometricOlver{\#1}{\#2}{\#3}{\#4}{\#5}$, the Olver's hypergeometric function with arguments.
    \item[\texttt{\textbackslash identity}] $\identity$, the identity operator.
    \item[\texttt{\textbackslash innerproduct[1]}] $\innerproduct{\#1}$, the inner product.
    \item[\texttt{\textbackslash Li}] $\Li$, the polylogarithm function.
    \item[\texttt{\textbackslash pFq[2]}] $\pFq{\#1}{\#2}$, the hypergeometric function.
    \item[\texttt{\textbackslash pFqOlver[2]}] $\pFqOlver{\#1}{\#2}$, the hypergeometric function.
    \item[\texttt{\textbackslash sech}] $\sech$, the hyperbolic secant.
    \item[\texttt{\textbackslash sign}] $\sign$, the sign function.
    \item[\texttt{\textbackslash spanspace}] $\spanspace$, the span space.
    \item[\texttt{\textbackslash supp}] $\supp$, the function support.
    \item[\texttt{\textbackslash tceil[1]}] $\tceil{\#1}$, rounding up with tiny delimiter.
    \item[\texttt{\textbackslash tfloor[1]}] $\tfloor{\#1}$, rounding down with tiny delimiter.
\end{itemize}

\section{Theorem environments}
\begin{theorem}
    This is a theorem.
\end{theorem}
\begin{corollary}
    This is a corollary.
\end{corollary}
\begin{proposition}
    This is a proposition.
\end{proposition}
\begin{lemma}
    This is a lemma.
\end{lemma}
\begin{conjecture}
    This is a conjecture.
\end{conjecture}

\begin{remark}
    This is a remark.
\end{remark}

\begin{example}
    This is an example.
\end{example}

\section{Unicode characters}
⬯✗□×

\nocite{*}
\bibliographystyle{plain}
\bibliography{references}

\end{document}
