\documentclass{article}
\usepackage[utf8]{inputenc}

\usepackage{mathtools}
\usepackage{amsmath}
\usepackage{amsfonts}

\newcommand{\Beta}{\mathrm{B}} % Upper beta
\DeclareMathOperator{\sign}{sgn} % sign function
\DeclareMathOperator{\diag}{diag} % diagonal matrix
\DeclareMathOperator{\spanspace}{span}
\DeclareMathOperator{\supp}{supp} % function support
\DeclareMathOperator{\Li}{Li} % polylogarithm function
\DeclareMathOperator{\erfc}{erfc} % complementary error function
\DeclareMathOperator{\sech}{sech} % hyperbolic secant
\DeclareMathOperator{\csch}{csch} % hyperbolic cosecant
\DeclareMathOperator{\arsinh}{arsinh} % inverse hyperbolic sine
\DeclareMathOperator{\arcosh}{arcosh} % inverse hyperbolic cosine
\DeclareMathOperator{\artanh}{artanh} % inverse hyperbolic tangent
\DeclareMathOperator{\arcoth}{arcoth} % inverse hyperbolic cotangent
\DeclareMathOperator{\arsech}{arsech} % inverse hyperbolic secant
\DeclareMathOperator{\arcsch}{arcsch} % inverse hyperbolic cosecant
\AtBeginDocument{\renewcommand{\d}{\:\mathrm{d}}} % differential symbol
\newcommand{\pFq}[2]{{}_{#1}F_{#2}} % hypergeometric function
\newcommand{\half}[1][1]{\frac{#1}{2}} % lazy wrapping
\newcommand{\f}[1]{\!\left({#1}\right)\!} % lazy wrapping for ()
\newcommand{\abs}[1]{\left\lvert{#1}\right\rvert} % absolute value
\newcommand{\floor}[1]{\left\lfloor{#1}\right\rfloor} % rounding down
\newcommand{\ceil}[1]{\left\lceil{#1}\right\rceil} % rounding up
\newcommand{\tfloor}[1]{\lfloor{#1}\rfloor} % rounding down with tiny delimiter
\newcommand{\tceil}[1]{\lceil{#1}\rceil} % rounding up with tiny delimiter
\newcommand{\Atop}[2]{\genfrac{}{}{0pt}{}{#1}{#2}} % safe wrapping of \atop
\newcommand{\hypergeometric}[5]{\pFq{#1}{#2}\left(\Atop{#3}{#4};{#5}\right)} % hypergeometric function
\newcommand{\lazysplit}[1]{\[\begin{split}#1\end{split}\]}
\newcommand{\innerproduct}[1]{\left\langle{#1}\right\rangle} % inner product

% principal value integral
% credit: https://tex.stackexchange.com/a/760/199226
\def\Xint#1{\mathchoice
{\XXint\displaystyle\textstyle{#1}}%
{\XXint\textstyle\scriptstyle{#1}}%
{\XXint\scriptstyle\scriptscriptstyle{#1}}%
{\XXint\scriptscriptstyle\scriptscriptstyle{#1}}%
\!\int}
\def\XXint#1#2#3{{\setbox0=\hbox{$#1{#2#3}{\int}$ }
\vcenter{\hbox{$#2#3$ }}\kern-.6\wd0}}
\def\ddashint{\Xint=}
\def\dashint{\Xint-}

% combine multicols and subequations environments
% credit: https://tex.stackexchange.com/a/513552/199226
\newenvironment{multicolsubequations}[1]{
    \begin{subequations}
        \setlength{\abovedisplayskip}{0pt}
        \setlength{\belowdisplayskip}{0pt}
        \begin{multicols}{#1}
            \vspace*{-2\baselineskip}
}{
        \end{multicols}
    \end{subequations}
}

% add an extra tag to the left of an equation
% credit: https://tex.stackexchange.com/a/664316/199226
\makeatletter
\def\ltag#1{%
\stepcounter{equation}%
\tag*{}%
\def\df@tag{(\theequation)\llap{\rlap{#1}\hspace{\columnwidth}}}%
}
\makeatother
\usepackage{amsthm}
\newtheorem{theorem}{Theorem}
\newtheorem{corollary}{Corollary}
\newtheorem{proposition}{Proposition}
\newtheorem{lemma}{Lemma}
\newtheorem{conjecture}{Conjecture}

\theoremstyle{remark}
\newtheorem{remark}{Remark}

\theoremstyle{definition}
\newtheorem{definition}{Definition}
\newtheorem{example}{Example}
\newtheorem{calculation}{Calculation}

\title{Preambles}
\author{Tianyi Pu}
\date{\today}

\begin{document}

\maketitle

\section{Commands}
\begin{itemize}
    \item[\texttt{\textbackslash abs[1]}] $\abs{\#1}$, the absolute value.
    \item[\texttt{\textbackslash arcosh}] $\arcosh$, the inverse hyperbolic cosine.
    \item[\texttt{\textbackslash arcoth}] $\arcoth$, the inverse hyperbolic cotangent.
    \item[\texttt{\textbackslash arcsch}] $\arcsch$, the inverse hyperbolic cosecant.
    \item[\texttt{\textbackslash arsech}] $\arsech$, the inverse hyperbolic secant.
    \item[\texttt{\textbackslash arsinh}] $\arsinh$, the inverse hyperbolic sine.
    \item[\texttt{\textbackslash artanh}] $\artanh$, the inverse hyperbolic tangent.
    \item[\texttt{\textbackslash Atop[2]}] $\Atop{\#1}{\#2}$, a safe wrapping of \texttt{\textbackslash atop}, the line-less fraction.
    \item[\texttt{\textbackslash Beta}] $\Beta$, the big $\beta$.
    \item[\texttt{\textbackslash csch}] $\csch$, the hyperbolic cosecant.
    \item[\texttt{\textbackslash d}] $\d$, the differential operator.
    \item[\texttt{\textbackslash diag}] $\diag$, a diagonal matrix.
    \item[\texttt{\textbackslash erfc}] $\erfc$, the complementary error function.
    \item[\texttt{\textbackslash f[1]}] $\f{\#1}$, a lazy wrapping of \texttt{\textbackslash left(\#1\textbackslash right)}.
    \item[\texttt{\textbackslash half[1][1]}] $\half[\#1]$, a lazy wrapping of halves.
    \item[\texttt{\textbackslash hypergeometric[5]}] $\hypergeometric{\#1}{\#2}{\#3}{\#4}{\#5}$, the hypergeometric function with arguments.
    \item[\texttt{\textbackslash innerproduct[1]}] $\innerproduct{\#1}$, the inner product.
    \item[\texttt{\textbackslash Li}] $\Li$, the polylogarithm function.
    \item[\texttt{\textbackslash pFq[2]}] $\pFq{\#1}{\#2}$, the hypergeometric function.
    \item[\texttt{\textbackslash sech}] $\sech$, the hyperbolic secant.
    \item[\texttt{\textbackslash sign}] $\sign$, the sign function.
\end{itemize}

\section{Theorem environments}
\begin{theorem}
    This is a theorem.
\end{theorem}
\begin{corollary}
    This is a corollary.
\end{corollary}
\begin{proposition}
    This is a proposition.
\end{proposition}
\begin{lemma}
    This is a lemma.
\end{lemma}
\begin{conjecture}
    This is a conjecture.
\end{conjecture}

\begin{remark}
    This is a remark.
\end{remark}

\begin{example}
    This is an example.
\end{example}

\nocite{*}
\bibliographystyle{plain}
\bibliography{references}

\end{document}
